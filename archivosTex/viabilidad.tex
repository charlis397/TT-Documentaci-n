\chapter{Estudio de viabilidad} % (fold)


\section{Producto. }

Un sistema que genere y almacene energía atreves de una celda solar y la irradiación que produce el Sol. El sistema se compone de un Mecanismo de generación de energía eléctrica responsable de seguir la trayectoria del Sol (similar al comportamiento de un girasol) para generar la mayor cantidad de energía eléctrica; además de una aplicación móvil que permite mostrar información del estado y funcionamiento de la celda solar y la batería.

\section{Mercado. }

Habitantes de la Ciudad de México con un rango de edad de entre 24 años y 50 años, que cuenten con Smartphone ya que en éste grupo se concentran las personas que pueden pagar por el producto y tienen la experiencia en el uso de Smartphones (Expansión, 2018); además de ser el grupo que más conciencia ambiental tiene (López Villarreal, Solís López, & Pérez Nieva, 2008).


%\section{Plan económico financiero. }




\section{Análisis FODA.}

\begin{table}[!htb]
\begin{tabular}{|p{2cm}|p{6cm}|p{6cm}|}
    \hline
     & \centering Fortalezas & \multicolumn{1}{c|}{\centering Debilidades} \\
    \hline
    \centering Análisis interno & \parbox[p][0.6\textwidth][c]{5.5cm}{
    \begin{itemize}
        \item El conocimiento en las áreas que permiten desarrollar un sistema de esta índole.  
        \item Experiencia desarrollando proyectos con características similares al del sistema.
        \item Descuentos por parte de proveedores.
    \end{itemize} } & \parbox[p][0.6\textwidth][c]{5.5cm}{
    \begin{itemize}
        \item Incertidumbre en el financiamiento del proyecto.
        \item Capital humano justo, sin oportunidad a alguna baja.
    \end{itemize} }  \\
    \hline
    & \centering Oportunidades & \hfil Amenazas \hfil\\
    \hline
    \centering Análisis externo & \parbox[p][0.6\textwidth][c]{5.5cm}{
    \begin{itemize}
        \item La mayoría de los habitantes de la CDMX usan Smartphones.
        \item Los habitantes de la CDMX están preocupados por el medio ambiente.
        \item La irradiación solar en la CDMX es alta.
        \item México se encuentra en el Acuerdo de París.
    \end{itemize} } & \parbox[p][0.6\textwidth][c]{5.5cm}{
    \begin{itemize}
        \item  Gran cantidad de energías renovables que existen.
        \item  La estructura de las viviendas en la CDMX complica la instalación de paneles solares.
        \item  Condiciones económicas de los habitantes del a CDMX.
    \end{itemize} }   \\
    \hline
\end{tabular}
\caption{Análisis de FODA.}
\label{tabla:foda}
\end{table}

\begin{itemize}
    \item Estrategia FO
    \begin{itemize}
         \item Utilizar la experiencia y el conocimiento en el desarrollo de sistemas para la creación de una aplicación móvil que permita mostrar las condiciones en las que se encuentra el mecanismo de obtención de energía para personas de la CDMX que utilizar smartphone
    \end{itemize}
    \item Estrategias DO
    \begin{itemize}
         \item Aprovechamiento de descuentos de proveedores para el financiamiento del proyecto. Alianzas estratégicas con proveedores para financiar el desarrollo del proyecto.
          \item Contar con la ayuda de profesores para el desarrollo del sistema ante el bajo capital humano.
    \end{itemize}
    \item Estrategias FA
     \begin{itemize}
        \item
    \end{itemize}
    \item Estrategias DA
     \begin{itemize}
         \item Aprovechamiento de los recursos económicos para desarrollar la aplicación móvil libre de costo para los consumidores (habitantes de la CDMX)
    \end{itemize}
\end{itemize}











