\chapter{Introducci\'on} % (fold)

%\section{Planteamiento del problema.}
El mundo entero es una sociedad de creciente expansión donde la necesidad de electricidad es indispensable para el progreso de un país. La humanidad ya ha explotado casi el 40\% de todos los yacimientos petrolíferos del planeta. En cuanto al calentamiento global, es bien sabido que el creciente contenido de dióxido de carbono en la atmósfera es principal responsable del cambio climático. \\
 
México camina hacia un colapso energético, lo cual nos obliga a buscar nuevas alternativas para dejar de ser
dependientes del uso de hidrocarburos para la generación de energía eléctrica. México es un país con alta incidencia de energía solar en la gran mayoría de su territorio; la zona norte es de las más soleadas del mundo ~\ref{fig:1-1-1}. La mitad del territorio nacional presenta una insolación promedio de 5.3 KWh por metro cuadrado al día, suficiente para satisfacer la necesidad de un hogar mexicano promedio. Esto nos coloca en una situación muy favorable para el uso de la energía solar. [1] La irradiación promedio diaria cambia a lo largo de la república y depende también del mes en cuestión, descendiendo ligeramente por debajo de 3 KWh por metro cuadrado al día y pudiendo alcanzar valores superiores a 8.5 KWh por metro cuadrado al día. [2] \\


En el año 2015 se llevó a cabo la Conferencia de París sobre el Cambio Climático, donde 195 países (incluyendo México)
alcanzaron acuerdos y compromisos para mejorar la situación ambiental en el planeta. Donde se resalta el artículo 10 que aborda el tema “Desarrollo y transferencia de tecnología”; en su apartado 4° y 5° alienta a promover y facilitar el desarrollo tecnológico. En el documento de los Acuerdos de París en el artículo 10 tratan el tema de Desarrollo y transferencia de tecnología.
'Impartirá orientación general al Mecanismo Tecnológico en su labor de promover y facilitar el fortalecimiento del desarrollo y la transferencia de tecnología a fin de respaldar la aplicación del presente Acuerdo'. [4]
La Secretaría de Energía ha fijado la meta de incrementar la participación de energías limpias en la de generación eléctrica a 35\% del total para el 2026 [5] con diferentes líneas de acción, como lo son:


\begin{itemize}
    \item Promover la inversión en sistemas fotovoltaicos en zonas del país con alto potencial
    \item Fomentar la generación distribuida mediante el uso de sistemas fotovoltaicos, en el sector industrial, residencia y de servicios
\end{itemize}

\begin{figure}[H]
\centering
%\includegraphics[width=13cm, height=5cm]{./images/mapaMundial.jpg}
\caption{Mapa mundial de la radiación solar}
\label{fig:1-1-1}
\end{figure}


\section{Propuesta de solución. }

Desarrollar un sistema encargado de generar y almacenar la energía a partir de una celda fotovoltaica que para maximizar la producción de la corriente eléctrica tendrá un movimiento que asemeje al de un girasol, el sistema podrá determinar si la batería donde se guarda la energía producida puede almacenar más energía. Está información (energía almacenada, estado de la batería, corriente eléctrica que genera la celda) podrá ser visualizada y monitoreada por medio de una aplicación móvil que además tiene la función de mostrar noticias, guía de instalación y determinar la orientación que debe tener una celda solar a partir de la estación del año y la ubicación geográfica.





\section{Objetivos. } % (fold)
A continuaci\'on se describir\'an los objetivos que se pretende alcanzar, durante el desarrollo del presente trabajo terminal. 

\subsection{Objetivo General.} % (fold)
Desarrollar un sistema encargado del monitoreo de la producción de corriente eléctrica que tiene un panel fotovoltaico por
medio de una aplicación móvil, con el propósito de reducir la producción de energía por medio de los hidrocarburos.
\subsection{Objetivos Específicos.} % (fold)
\begin{itemize}
\item Desarrollar un mecanismo que permita situar un panel fotovoltaico con la inclinación donde se produce la mayor
cantidad de corriente eléctrica.
\item Desarrollar un algoritmo que determine la fase en la que se encuentra la batería que almacena la energía, y con ello determinar si puede almacenar más energía o si está llena.
\item Desarrollar una aplicación móvil, que permita mostrar la inclinación ideal de un panel solar en México en función de la estación del año y la ubicación geográfica.
\item Implementar una interfaz que comuniqué el mecanismo de generación de energía con la aplicación móvil, para que ésta última reciba los datos proporcionados por el mecanismo. 
\item La aplicación móvil proporcionará información sobre la instalación de un panel solar, su funcionamiento y noticias de energías verdes.
\item La aplicación móvil mostrará el rendimiento que tenga el mecanismo de generación de energía a lo largo del día, semanalmente o mensualmente.

\end{itemize}

\section{Justificación. }


El Sol es una fuente de energía que hace pocos años se comenzó a aprovechar, diversos expertos en astronomía y científicos han trabajado en conjunto para encontrar un buen uso de la energía proveniente de la radiación solar. La energía solar es una gran fuente de energía para llenar nuestras necesidades. Aparte de la que nos llega directamente de sus rayos, el Sol también es el origen de otras fuentes de energía. Por ejemplo, el viento es causado por las diferencias de temperatura en distintos lugares del mundo, y la energía hidráulica depende del ciclo hidrológico, el cual tiene su origen en la evaporación de las aguas causada por el Sol. Así mismo, los combustibles fósiles (petróleo y carbón, entre otros) se produjeron gracias a la energía transmitida por este astro, luego de transformarse a partir de su forma vegetal.
 ~\cite{Keelveedhi}. \\
 
Considerando la capacidad energética del Sol, la cual perdurará durante millones de años, así como la privilegiada ubicación de
México en el globo terráqueo, la cual permite que el territorio nacional destaque en el mapa mundial de territorios con mayor
promedio de radiación solar anual, por esta razón resulta fundamental el uso de las tecnologías que tenemos hoy en día para
aprovechar esta situación.
La energía solar es un recurso renovable prácticamente ilimitado. Hay virtualmente una provisión ilimitada de energía solar que
podemos usar y es una energía renovable. Esto significa que nuestra dependencia de combustibles fósiles se puede reducir en
proporción directa a la cantidad de energía solar que producimos. Con el constante incremento en la demanda de fuentes de energía
tradicionales y el consiguiente aumento en los costos, la energía solar es cada vez más una necesidad, es una excelente fuente de
energía alternativa porque no hay contaminación al usarse.
Un sistema de energía solar para generación eléctrica en el hogar puede potencialmente eliminar hasta 18 toneladas de emisiones
de gases de invernadero al ambiente cada año. \\








